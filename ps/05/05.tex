\documentclass{../../e85}

\ps{5}
\date{2019}
\author{}
\coauthor{}

\begin{document}
\begin{enumerate}
\item SystemVerilog to State Transition Diagram

  Do Exercise 4.24 from DDCA ARMed.

  \begin{description}
  \item[4.24] Sketch the state transition diagram for the FSM
    described by the following HDL code.
    \begin{center}
      \twocolumnlisting
      {SystemVerilog}{systemverilog}{code/ddca-4.24.sv}
      {VHDL}{vhdl}{code/ddca-4.24.vhdl}
    \end{center}

    \begin{solution}
    \end{solution}
  \end{description}

\item SystemVerilog Logic Design

  Design a serial (one bit at a time) two's complementer FSM with two
  inputs, \(\mathrm{Start}\) and \(A\), and one output, \(Q\).
  \(\mathrm{Start}\) is asserted to initialize the FSM before the
  least significant bit is provided.  In other words, develop the
  machine that performs the two's complement operation on a binary
  number of arbitrary length, starting with the least significant bit.
  A one-bit input \(A\) simultaneously produces a one-bit output
  \(Q\).  For example, given \mintinline{SystemVerilog}|4'b0010|,
  \(A\) receives `\(0\)' then `\(1\)', `\(0\)', `\(0\)', respectively
  and the FSM outputs `\(0\)', `\(1\)', `\(1\)', `\(1\)' with the
  corresponding output bit appearing at \(Q\) on the same cycle.
  Express your design in behavioral SystemVerilog. \\
  \ul{Hint}: You may find it helpful to try some more cases by hand to
  figure out how the output should \emph{depend} on the input.  You
  may also want to look at Example 3.7 in Harris \& Harris.

  \begin{solution}
  \end{solution}

\item Priority Circuit

  An \(N\)-input lowest priority circuit has input \(A_{N-1:0}\) and
  output \(Y_{N-1:0}\).  \(Y_j = 1\) only if \(A_k = 0\) for all
  \(k<j\) and \(A_j = 1\).  In other words, it asserts a bit in \(Y\)
  corresponding to the \emph{least} significant bit asserted in \(A\).

  Design a \(16\)-input lowest priority circuit with a propagation
  delay not exceeding \SI{55}{\ps}.  Suppose each \(2\)-input gate has
  a delay of \SI{10}{\ps} and an inverter has a delay of \SI{5}{\ps}.
  Sketch a schematic for your circuit.  If you are spending too much
  time on this problem, you may relax the delay requirement to
  \SI{180}{\ps} for partial credit.  How would you design it?  Then,
  increase the input size and look for any pattern.

  \begin{solution}
  \end{solution}

\item Impact on society: Identify a product that you use directly or
  indirectly in your life that contains an FPGA.  Why would the
  designers choose to use an FPGA in that product?

  \begin{solution}
  \end{solution}
\end{enumerate}

How long did you spend on this problem set?  This will not count
toward your grade but will help calibrate the workload.
\begin{solution}
\end{solution}
\end{document}
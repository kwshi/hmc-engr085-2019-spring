\documentclass{e85}

\ps{3}
\date{2019 January}
\author{}
\coauthor{}

\begin{document}

\begin{enumerate}
\item Suppose propagation delays for a \SI{10}{\nm} integrated circuit
  process are given in the table below.
  \begin{center}
    \begin{tabular}{c|cc}
      \textbf{Cell}
      &\textbf{Propagation Delay (\si{\ps})}
      &\textbf{Contamination Delay (\si{\ps})}\\\midrule
      NOT  &6 &4\\
      NAND2&8 &6\\
      NOR2 &10&8\\
      NAND3&10&8\\
      NOR3 &12&10
    \end{tabular}
  \end{center}
  Determine the propagation and contamination delays of the following
  circuit.  Redesign it to accomplish the same function while
  minimizing propagation delay, using only gates from the table above.
  What are the propagation and contamination delays of your optimized
  circuit?
  \begin{center}
    \begin{tikzpicture}[circuit logic US, x=.5cm, y=.5cm, huge circuit symbols]
      \draw

      (0,0) node[nand gate, inputs=nnn](1-nand){}
      (1-nand.output) node[right=1, buffer gate, inputs=i] (1-not) {}

      let
      \p1 = ($(1-nand.input 1) - (1-nand.input 3)$)
      in
      (1-not.output) ++(0,-\y1) node[right=1, nand gate, inputs=nnn](2-nand){}
      (2-nand.output) node[right=1, buffer gate, inputs=i](2-not){}

      (2-not.output) ++(0,-\y1) node[right=1, nand gate, inputs=nnn](3-nand){}
      (3-nand.output) node[right=1, buffer gate, inputs=i](3-not){}

      (1-nand.output) -- (1-not.input)
      (2-nand.output) -- (2-not.input)
      (3-nand.output) -- (3-not.input)

      let
      \p2 = ($(1-not.output)!.5!(2-nand.input 1)$),
      \p3 = ($(2-not.output)!.5!(3-nand.input 1)$)
      in
      (1-not.output) -| (\p2) |- (2-nand.input 1)
      (2-not.output) -| (\p3) |- (3-nand.input 1)

      (1-nand.input 1) -- +(-.5,0) coordinate(A) node[left]{\(A\)}

      let
      \p4 = (A),
      \p5 = (1-nand.input 3),
      \p6 = (2-nand.input 3),
      \p7 = (3-nand.input 3)
      in
      (\p5) -- (\x4,\y5) node[left]{\(B\)}
      (\p6) -- (\x4,\y6) node[left]{\(C\)}
      (\p7) -- (\x4,\y7) node[left]{\(D\)}

      (3-not.output) -- +(.5,0) node[right]{\(Y\)};
    \end{tikzpicture}
  \end{center}
  \begin{solution}
  \end{solution}

\item Latches and Flip-Flops

  Do Exercises 3.4 and 3.6 from the textbook.
  \newcommand{\waveform}[1]{
    foreach \x/\y in {#1} {
      -- ++(\x,0) -- ++(.25,\y)
    }
  }
  \newcommand{\inputwaveforms}{
    \draw
    (0,0) node[above]{CLK}
    \waveform{
      1.25/1,
      1/-1,
      1.75/1,
      .625/-1,
      1.75/1,
      .875/-1,
      .5/1,
      2/-1,
      1.25/1
    }
    -- +(1.25,0)

    (0,-1.5) node[above]{D}
    \waveform{
      1.875/1,
      1/-1,
      .5/1,
      .875/-1,
      .875/1,
      1.375/-1,
      1.75/1,
      .375/-1,
      .375/1,
      2.125/-1
    }
    -- +(.875,0);
  }
  \begin{description}
  \item[3.4] Given the input waveforms shown in Figure 3.64 \figurebelow, sketch
    the output, \(Q\), of a D latch.
    \begin{center}
      \begin{tikzpicture}[scale=2.54/4]
        \inputwaveforms
      \end{tikzpicture}
    \end{center}
    \begin{solution}
    \end{solution}

  \item[3.6] Given the input waveforms shown in Figure 3.64
    \figurebelow, sketch the output, \(Q\), of a D flip-flop.
    \begin{center}
      \begin{tikzpicture}[scale=2.54/4]
        \inputwaveforms
      \end{tikzpicture}
    \end{center}
    \begin{solution}
    \end{solution}
  \end{description}

\item Combinational and Sequential Logic

  Do Exercise 3.18 from the textbook.
  \begin{description}
  \item[3.18] Which of the circuits in Figure 3.68 are synchronous
    sequential circuits?  Explain.
    \begin{center}
      \begin{tabu}{rl@{\qquad}rl}
        (a)&\includegraphics{figures/{ddca-3.68-a}.eps}&
        (b)&\includegraphics{figures/{ddca-3.68-b}.eps}\\[1em]
        (c)&\includegraphics{figures/{ddca-3.68-c}.eps}&
        (d)&\includegraphics{figures/{ddca-3.68-d}.eps}
      \end{tabu}
    \end{center}
    \begin{solution}
    \end{solution}
  \end{description}

\item FSM design

  Do Exercise 3.26 from the textbook.  Just sketch the state
  transition diagram.  Don't do the complete design that is requested
  in the problem.

  \textit{Extra Credit}: Finish the complete design.  Don't start on
  the extra credit until you've finished the rest of PS \theps.  You
  do have other classes, and all humans need sleep.
  \begin{description}
  \item[3.26] You have been enlisted to design a soda machine
    dispenser for your department lounge.  Sodas are partially
    subsidized by the student chapter of the IEEE, so they cost only
    25 cents.  The machine accepts nickels, dimes, and quarters.  When
    enough coins have been inserted, it dispenses the soda and returns
    any necessary change.  Design an FSM controller for the soda
    machine.  The FSM inputs are \textit{Nickel}, \textit{Dime},
    \textit{Quarter}, indicating which coin was inserted.  Assume that
    exactly one coin is inserted on each cycle.  The outputs are
    \textit{Dispense}, \textit{ReturnNickel}, \textit{ReturnDime},
    \textit{ReturnTwoDimes}.  When the FSM reaches 25 cents, it
    asserts \textit{Dispense} and the necessary return outputs
    required to deliver the appropriate change.  Then it should be
    ready to start accepting coins for another soda.
    \begin{solution}
    \end{solution}
  \end{description}

\item Impact on society: Name a system (other than a traffic light or
  soda machine dispenser) that you encounter in daily life that is
  readily described as a finite state machine.
  \begin{solution}
  \end{solution}
\end{enumerate}

How long did you spend on this problem set?  This will not count
toward your grade but will help calibrate the workload.
\begin{solution}
\end{solution}

\end{document}
\documentclass{ps}

\ps{3}
\date{2019 January}
\author{}
\coauthor{}

\begin{document}

\begin{enumerate}
\item Suppose propagation delays for a \SI{10}{\nm} integrated circuit
  process are given in the table below.
  \begin{center}
    \begin{tabular}{c|cc}
      \textbf{Cell}&\textbf{Propagation delay (\si{\ps})}&\textbf{Contamination delay (\si{\ps})}\\\midrule
      NOT&6&4\\
      NAND2&8&6\\
      NOR2&10&8\\
      NAND3&10&8\\
      NOR3&12&10
    \end{tabular}
  \end{center}
  Determine the propagation and contamination delays of the following
  circuit.  Redesign it to accomplish the same function while
  minimizing propagation delay, using only gates from the table above.
  What are the propagation and contamination delays of your optimized
  circuit?
  \begin{center}
    \begin{tikzpicture}[circuit logic US, x=1.5cm, y=.5cm, huge circuit symbols]
      \draw

      (0,0) node[nand gate, inputs=nnn](1-nand){}
      (1,0) node[buffer gate, inputs=i] (1-not) {}

      let
      \p1 = ($(1-nand.input 1) - (1-nand.input 3)$)
      in
      (2,-\y1) node[nand gate, inputs=nnn](2-nand){}
      (3,-\y1) node[buffer gate, inputs=i] (2-not) {}
      (4,{-2*\y1}) node[nand gate, inputs=nnn](3-nand){}
      (5,{-2*\y1}) node[buffer gate, inputs=i] (3-not) {}

      (1-nand.input 1) -- +(-.25,0) node[left]{\(A\)}
      (1-nand.input 3) -- +(-.25,0) node[left]{\(B\)}
      (2-nand.input 3) -- +(-2.25,0) node[left]{\(C\)}
      (3-nand.input 3) -- +(-4.25,0) node[left]{\(D\)}

      (1-nand.output) -- (1-not.input)

      let
      \p1 = ($(1-not.output)!.5!(2-nand.input 1)$),
      \p2 = ($(2-not.output)!.5!(3-nand.input 1)$)
      in
      (1-not.output) -| (\p1) |- (2-nand.input 1)
      (2-not.output) -| (\p2) |- (3-nand.input 1)

      (2-nand.output) -- (2-not.input)
      (3-nand.output) -- (3-not.input)

      (3-not.output) -- +(.25,0) node[right]{\(Y\)};
    \end{tikzpicture}
  \end{center}
  \begin{solution}
  \end{solution}

\item Latches and Flip-Flops

  Do Exercises 3.4 and 3.6 from the textbook.
  \begin{solution}
  \end{solution}

\item Combinational and Sequential Logic

  Do Exercise 3.18 from the textbook.
  \begin{solution}
  \end{solution}

\item FSM design

  Do Exercise 3.26 from the textbook.  Just sketch the state
  transition diagram.  Don't do the complete design that is requested
  in the problem.

  \textit{Extra Credit}: Finish the complete design.  Don't start on
  the extra credit until you've finished the rest of PS \theps.  You
  do have other classes, and all humans need sleep.
  \begin{solution}
  \end{solution}

\item Impact on society: Name two systems that you encounter in daily
  life that are readily described as finite state machines.
  \begin{solution}
  \end{solution}
\end{enumerate}

How long did you spend on this problem set?  This will not count
toward your grade but will help calibrate the workload.
\begin{solution}
\end{solution}

\end{document}
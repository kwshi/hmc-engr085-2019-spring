\documentclass{ps}

\ps{5}
\date{2019}
\author{}
\coauthor{}

\begin{document}
\begin{enumerate}
\item SystemVerilog to State Transition Diagram

  Do Exercise 4.24 from DDCA ARMed.

  \begin{description}
  \item[4.24] Sketch the state transition diagram for the FSM
    described by the following HDL code.
    \begin{center}
      \twocolumnlisting
      {SystemVerilog}{systemverilog}{code/ddca-4.24.sv}
      {VHDL}{vhdl}{code/ddca-4.24.vhdl}
    \end{center}
    \begin{solution}
    \end{solution}
  \end{description}

\item SystemVerilog Logic Design

  Design a serial (one bit at a time) two's complementer FSM with two
  inputs, \(\mathrm{Start}\) and \(A\), and one output, \(Q\).  A
  binary number of arbitrary length is provided to input \(A\),
  starting with the least significant bit.  The corresponding bit of
  the output appears at \(Q\) on the same cycle.  Start is asserted
  for one cycle to initialize the FSM before the least significant bit
  is provided.  You may find it helpful to try some cases by hand to
  figure out how the output should depend on the input.  Express your
  design in behavioral SystemVerilog.
  \begin{solution}
  \end{solution}

\item Priority Circuit

  An \(N\)-input lowest priority circuit has input \(A_{N-1:0}\) and
  output \(Y_{N-1:0}\).  \(Y_j = 1\) if \(A_j = 1\) and \(A_k = 0\)
  for all \(k<j\).  In other words, it asserts a bit in \(Y\)
  corresponding to the \emph{least} significant bit asserted in \(A\).

  Suppose each \(2\)-input gate has a delay of \SI{10}{\ps} and an
  inverter has a delay of \SI{5}{\ps}.  Design a \(16\)-input lowest
  priority circuit with a propagation delay not exceeding
  \SI{55}{\ps}.  Sketch a schematic for your circuit.  If you are
  spending too much time on this problem, you may relax the delay
  requirement to \SI{180}{\ps} for partial credit.
  \begin{solution}
  \end{solution}

\item Impact on society: Identify an item you use in your life that
  you know or could reasonably expect was primarily implemented by
  synthesizing hardware description language code onto a chip or FPGA
  (as opposed to mainly writing code for a microcontroller, building
  an analog electronic or mechanical device, etc.)  Why was HDL an
  appropriate design methodology for the item?
  \begin{solution}
  \end{solution}
\end{enumerate}

How long did you spend on this problem set?  This will not count
toward your grade but will help calibrate the workload.
\begin{solution}
\end{solution}
\end{document}
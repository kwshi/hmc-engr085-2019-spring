\documentclass{ps}

\ps{4}
\date{2019}
\author{}
\coauthor{}

\begin{document}

\begin{enumerate}
\item Reconsider the circuit from PS3, with registers added.  What is
  the minimum clock period for the circuit in the absence of clock
  skew?  How much clock skew could the circuit endure before possibly
  violating the hold time?
  \begin{center}
    \begin{tabular}{c|cccc}
      \textbf{Cell}
      &\textbf{\makecell{Propagation\\Delay (\si{\ps})}}
      &\textbf{\makecell{Contamination\\Delay (\si{\ps})}}
      &\textbf{\makecell{Setup\\Time (\si{\ps})}}
      &\textbf{\makecell{Hold\\Time (\si{\ps})}}\\\midrule
      NOT  &6 &4 \\\hline
      NAND2&8 &6 \\\hline
      NOR2 &10&8 \\\hline
      NAND3&10&8 \\\hline
      NOR3 &12&10\\\hline
      Flop &20&15&10&5
    \end{tabular}

    \begin{tikzpicture}[circuit logic US, huge circuit symbols, x=.5cm, y=.5cm]
      \draw
      (0,0) node[nand gate, inputs=nnn](1-nand){}
      (1-nand.output) node[right=1, buffer gate, inputs=i] (1-not) {}

      let
      \p1 = ($(1-nand.input 1) - (1-nand.input 3)$)
      in
      (1-not.output) ++(0,-\y1) node[right=1, nand gate, inputs=nnn](2-nand){}
      (2-nand.output) node[right=1, buffer gate, inputs=i](2-not){}

      (2-not.output) ++(0,-\y1) node[right=1, nand gate, inputs=nnn](3-nand){}
      (3-nand.output) node[right=1, buffer gate, inputs=i](3-not){}

      (1-nand.output) -- (1-not.input)
      (2-nand.output) -- (2-not.input)
      (3-nand.output) -- (3-not.input)

      let
      \p2 = ($(1-not.output)!.5!(2-nand.input 1)$),
      \p3 = ($(2-not.output)!.5!(3-nand.input 1)$)
      in
      (1-not.output) -| (\p2) |- (2-nand.input 1)
      (2-not.output) -| (\p3) |- (3-nand.input 1)

      (1-nand.input 1) -- +(-1.5,0) coordinate(A)

      let
      \p4 = (A),
      \p5 = (1-nand.input 3),
      \p6 = (2-nand.input 3),
      \p7 = (3-nand.input 3)
      in
      foreach \i in {5,6,7} {
        (\p\i) -- (\x4,\y\i)
      }

      (3-not.output) -- +(1.5,0);

      \draw[fill=white]
      let
      \p1 = ($(1-nand.input 1) - (1-nand.input 3)$),
      \p2 = ($(A) + (0,-3*\y1) + (1,-.5)$)
      in
      (A) ++(.5,.5) rectangle (\p2)

      (3-not.output) ++(.5,.5) rectangle +(.5,-1);

      \newcommand{\clklabel}{
        ++(.5,.5) -- ++(.25,-.25) -- +(.25,.25)
        ++(0,.25) -- +(0,.5) node[above]{CLK}
      }
      \draw
      (A) \clklabel
      (3-not.output) \clklabel;
    \end{tikzpicture}
  \end{center}
  \begin{solution}
  \end{solution}

\item Synchronizers

  Do Exercise 3.36 from DDCA ARMed.
  \begin{description}
  \item[3.36] A synchronizer is built from a pair of flip-flops with
    \(t_\text{setup} = \SI{50}{\ps}\), \(T_0 = \SI{20}{\ps}\), and
    \(\tau = \SI{30}{\ps}\).  It samples an asynchronous input that
    changes \(10^8\) times per second.  What is the minimum clock
    period of the synchronizer to achieve a mean time between failures
    (MTBF) of 100 years?
  \end{description}
  \begin{solution}
  \end{solution}

\item SystemVerilog

  Do Exercises 4.2 and 4.13 from DDCA ARMed.  Use behavioral Verilog
  for 4.13.
  \begin{description}
  \item[4.2] Sketch a schematic of the circuit described by the
    following HDL code.  Simplify the schematic so that it shows a
    minimum number of gates.
    \begin{center}
      \twocolumnlisting
      {SystemVerilog}{systemverilog}{code/4.2.sv}
      {VHDL}{vhdl}{code/4.2.vhdl}
    \end{center}
    \begin{solution}
    \end{solution}
  \item[4.13] Write an HDL module for a 2:4 decoder.
    \begin{solution}
    \end{solution}
  \end{description}

\item Impact on society: What mean time between failure would you
  target for a synchronizer in a life-critical medical device?
  Justify your choice.  If the only synchronizer element readily
  available to you has a MTBF on \(10^9\) seconds, how could you
  achieve your target?
  \begin{solution}
  \end{solution}
\end{enumerate}

How long did you spend on this problem set?  This will not count
toward your grade but will help calibrate the workload.
\begin{solution}
\end{solution}

\end{document}
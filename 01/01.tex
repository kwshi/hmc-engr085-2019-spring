\documentclass{ps}

\ps{1}
\date{2019 January}
\author{}
\coauthor{}

\begin{document}

\begin{enumerate}
\item Number system finger exercise.  Don't use a calculator to
  automate these parts, though you can use it if you want help with
  multiplication or addition or to check answers.  If you find any
  parts difficult or time-consuming, make up some more problems until
  number systems feel easy.
  \begin{enumerate}
  \item Write the powers of \(2\) from \(2^0\) to \(2^{16}\).  Commit
    these numbers to memory because you will use them frequently in
    digital design.
    \begin{solution}
    \end{solution}
  \item Base conversions.  If you do these properly, you shouldn't
    have any difficult arithmetic when converting between bases \(2\), \(8\),
    and/or \(16\).
    \begin{enumerate}
    \item Convert the following numbers to base 2.\\
      \(13_{10}, 87_{10}, 1000_{10}, 5_8, 654_8, \mathrm{B}_{16},
      \mathrm{FEED}_{16}\)
      \begin{solution}
      \end{solution}
    \item Convert the following numbers to base 10.\\
      \(1001_2, 1100100_2, 5_8, 654_8, \mathrm{B}_{16},
      \mathrm{FEED}_{16}\)
      \begin{solution}
      \end{solution}
    \item Convert the following numbers to base 16.\\
      \(1001_2, 1100100_2, 5_8, 654_8, 17_{10}, 200_{10}, 1000_{10}\)
      \begin{solution}
      \end{solution}
    \end{enumerate}
  \item Number systems
    \begin{enumerate}
    \item Convert the following numbers to 8-bit 2's complement and
      sign-magnitude format:\\
      \(69_{10}, -2_{10}, -37_{10}\)
      \begin{solution}
      \end{solution}
    \item Convert the following 6-bit 2's complement numbers to base 10:\\
      \(100100_2, 011111_2\)
      \begin{solution}
      \end{solution}
    \item Convert the following 6-bit sign-magnitude numbers to base 10:\\
      \(100100_2, 011111_2\)
      \begin{solution}
      \end{solution}
    \item Write the most positive and most negative 8-bit numbers in
      binary and decimal for each of the following formats: unsigned,
      2's complement, sign-magnitude.
      \begin{solution}
      \end{solution}
    \end{enumerate}
  \item Arithmetic:
    \begin{enumerate}
    \item Assuming unsigned format:
      \begin{enumerate}
      \item Compute \(1010_2 + 0111_2\).  Convert the addends and the
        sum to decimal and check your results.
        \begin{solution}
        \end{solution}
      \item Extend \(101111_2\) to 8 bits properly.  Convert the input
        and result to decimal and check your result.
        \begin{solution}
        \end{solution}
      \end{enumerate}
    \item Repeat the question above assuming 2's complement format.
      \begin{solution}
      \end{solution}
    \end{enumerate}
  \end{enumerate}

\item Logic gates

  Write the symbol, Boolean equation, truth table, and Verilog code
  for a 3-input NAND gate.
  \begin{solution}
  \end{solution}

\item Impact on Society

  Research and write a paragraph about how digital systems have
  transformed another discipline of engineering within your lifetime.
  Provide quantitative data about at least one specific impact, such
  as cost, reliability, etc.  Cite your sources.
  \begin{solution}
  \end{solution}

\end{enumerate}

How long did you spend on this problem set?  This will not count
toward your grade but will help calibrate the workload.
\begin{solution}
\end{solution}

\end{document}